\section{凸解析}
%

%
\subsection{関数の増加と減少}
\paragraph{狭義単調増加}
関数$f$が
 \begin{align}
 x_1<x_2\to f(x_1)<f(x_2)
\end{align}
を満たすとき,$f$は狭義単調増加するという.
\paragraph{狭義単調減少}
関数$f$が
 \begin{align}
 x_1<x_2\to f(x_1)>f(x_2)
\end{align}
を満たすとき,$f$は狭義単調減少するという.
\paragraph{単調増加}
関数$f$が
 \begin{align}
 x_1<x_2\to f(x_1)\leq f(x_2)
\end{align}
を満たすとき,$f$は単調増加するという.また,$f(x)=$定数も増加関数である.



\paragraph{単調減少}
関数$f$が
 \begin{align}
 x_1<x_2\to f(x_1)\geq f(x_2)
\end{align}
を満たすとき,$f$は単調減少するという.





\section{凸関数}
\subsection{1変数の凸関数}
$f$を閉区間$[a,b]$で定義された連続な関数とする.また$a<x<b$において$f^\prime$および$f^{\prime\prime}$が存在すると仮定する.次の
\begin{kotak}
	\begin{theorem}[増減の判定条件]\label{thm1}
	閉区間$[a,b]$における連続関数$f(x)$が,$a<x<b$なるすべての$x$において微分可能であるとする.このとき,
	\begin{enumerate}
	\item すべての$x$$(a<x<b)$で$f^\prime(x)>0$ならば,$f(x)$は$[a,b]$において増加関数である.
	\item すべての$x$$(a<x<b)$で$f^\prime(x)<0$ならば,$f(x)$は$[a,b]$において減少関数である.
 \end{enumerate}
\end{theorem}
\end{kotak}
より$f^\prime(x)$の条件における$f(x)$の増減の判定条件がわかる.\\
 それでは,すべての$x(a<x<b)$で$f^{\prime\prime}$が正の値をとるときに,曲線$y=f(x)$はどのような変化の様子となるか調べる.定理\ref{thm1}より$f^\prime(x)$は増加関数となる.\\
 $f^\prime(x)$は曲線$y=f(x)$の点$(x,f(x))$のおける接線の傾きであるから,$x$が増加すると傾きも増加するので,図\ref{}のように曲線が上向きに湾曲していると考えられる.\\
 この「上向きに湾曲している」というのは次のことを意味する.すなわち,「曲線上に任意に2点$P,Q$をとると,$P$と$Q$とを結ぶ割線$PQ$が$P$と$Q$を結ぶ曲線の上側に必ずある」ということである.このような場合にわれわれは,曲線$y=f(x)$が上に凹(concave upwards),または下に凸(conve downwards)であるといい,関数$f$を凸関数であると呼ぶ.
%
 \begin{figure}[H]
 \centering
\begin{tikzpicture}
 \draw[->,>=stealth,very thick] (-0.5,0)--(5.0,0)node[above]{}; %x軸
 \draw[->,>=stealth,very thick] (0,-0.5)--(0,5.0)node[right]{}; %y軸
 \draw (0,0)node[above left]{O}; %原点
 \draw[black,samples=100,domain=0.5:5] plot(\x,{(0.8*\x-2)^2+2.0})node[right]{$f(x)$};
 \draw[dashed] (pi*0.5,2.6)node[left]{P}--(4,3.4)--(4,0)node[below]{$x_1$}; %点(\pi,2)
  \draw[dashed] (4,3.4)node[right]{Q}--(pi*0.5,2.6)--(pi*0.5,0)node[below]{$x_2$}; %点(\pi,2)
\end{tikzpicture}
  \label{sin}
  \caption{$\sin\theta$のグラフ.($0\leq\theta\leq\pi$)}
\end{figure}











\subsection{}






