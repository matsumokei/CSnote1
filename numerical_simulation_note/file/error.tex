\part{}


\section{2種類の誤差}
まず,数値計算を行う際に生ずる誤差について説明する.主に以下の2種類の誤差がある.
\begin{kotak}
	\begin{description}
   \item[丸め誤差:]コンピュータが実数を有限の小数に変換する際に生じる誤差.
   \item[打ち切り誤差:]級数など無限に続く式や極限をとる操作を,計算のために途中で打ち切ったときに生ずる誤差.
\end{description}
\end{kotak}
本書では,この打ち切り誤差について考えることにする.

\section{オーダー記法について}
$\mathcal{O}$記法は,$n$の値を増やしたときに,関数$f(n)$の値がどの程度の勢いで増えていくかを表現するものである.増加率の高い項のみを残し,