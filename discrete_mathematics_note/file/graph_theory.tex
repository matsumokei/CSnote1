\part{グラフの基礎概念}

\section{基本的な定義}
\begin{definition}[グラフ]
グラフ$G$とは\textbf{頂点集合}$V(G)$と\textbf{辺集合}$E(G)$からなる図形のことである.頂点集合$V(G)$とは\textbf{頂点 (vertex, site)}あるいは\textbf{点}と呼ばれる要素の集合である.また,辺集合$E(G)$とは\textbf{辺 (edge, bond)}と呼ばれる頂点の2元集合の集合からなるものである.
辺$e\in E(G)$は,頂点$u$, $v$を用いて$e={u, v}$, $e=uv=vu$で表される.ここでは,$e={u, v}$で辺を表す.

頂点集合,辺集合がともに有限集合であるグラフを\textbf{有限グラフ}という.グラフ$G$の頂点数$|V(G)|$を$G$の\textbf{位数 (order)},辺数$|E(G)|$を$G$のサイズ (size)という.
\end{definition}
グラフには,無向グラフと有効グラフの2種類がある.
\begin{definition}[無向グラフ]
    辺が,節点2要素からなる集合$\{u,\ v\}$で表されるとき,$G$は\textbf{無向グラフ (undirected graph)}であるという.無向グラフは向きのないグラフであり,単にグラフと呼ぶ.
\end{definition}

\begin{definition}[有向グラフ]
    辺に向きのあるグラフ$D$は,\textbf{有向グラフ (directed graph)}と呼ばれ,頂点集合$V(D)$と\textbf{弧 (有向辺)}と呼ばれる頂点の順序対$\{u,v\}$の集合$A(D)$ (\textbf{弧集合})からなる.
\end{definition}

\begin{definition}[端点,隣接,接続]
グラフ$G$の頂点$u$と$v$が辺$e$で結ばれているとき,頂点$u$と$v$とは\textbf{隣接 (adjacent)}するという.辺$e$は頂点$u,\ v$と\textbf{接続する}といい,頂点$u,\ v$を辺$e$の\textbf{端点 (edge)}という.
また,頂点$v$に隣接する頂点全体の集合$v$の近傍といい,$N_G(v)$で表す.辺$e_1$と$e_2$が同一の頂点$u$に接続しているとき2辺は隣接するという.
\end{definition}

グラフ$G$のある接点$v$に接続する辺の個数を\textbf{次数 (degree)}といい,$\text{deg}_G(v)$または$G$が明らかなときには単に$\text{deg}(v)$と表す.
有向グラフの頂点の次数は,入次数 (いりじすう,in-degree)と出次数 (でじすう,out-degree)に分けて定義される.
\begin{definition}[入次数,出次数]
    $G$が有向グラフであるとき,節点$v$へ入る辺の個数を\textbf{入次数 (in-degree)}といい,$\text{deg}_G^{\text{in}}(v)$または,$\text{deg}^{\text{in}}(v)$と表す.一方,節点$v$から出る辺の個数を\textbf{出次数 (out-degree)}といい,$\text{deg}_G^{\text{out}}(v)$または,$\text{deg}^{\text{out}}(v)$と表す.明らかに,$\text{deg}(v)=\text{deg}^{\text{in}}(v)+\text{deg}^{\text{out}}(v)$であり,辺の個数が保存していることがわかる.
\end{definition}



\subsection{木構造}
\begin{definition}
    $G$が無閉路 (閉路を含まない) グラフは森,林という.また,$G$が無閉路かつ連結グラフであるとき,$G$を\textbf{木 (tree)}という.特に,有向グラフ$G$が木であるとき,$G$を\textbf{有向木 (directed tree)}という.
\end{definition}

\begin{definition}[根付き木,二分木]
    有向木 $T=(V,E)$が一つの接点$r\in V$について,入次数が0であり,その他の各$v\in V-\{r\}$について入次数が1であるとき,$T$を\textbf{根付き木 (rooted tree)} といい,$r$を\textbf{根 (root)}という.(これは入次数0の頂点をちょうど1個もち,ほかの頂点の入次数がすべて1であるということである.) 各節点 (ノード点) の出次数が高々2である根付き木のことを\textbf{二分木 (binary tree)}という. 
\end{definition}